\documentclass[a4paper]{easychair}

\title{Compact, totally separated, and well-ordered types in univalent mathematics}
\author{Mart{\'\i}n H\"otzel Escard\'o}
\authorrunning{Mart{\'\i}n H\"otzel Escard\'o}
\titlerunning{Compact, totally separated, and well-ordered types}
\institute{School of Computer Science, Universe of Birmingham, UK \\ \email{m.escardo@cs.bham.ac.uk}}

\usepackage{amssymb,bbm}
\usepackage{hyperref}
\usepackage{xcolor}

\definecolor{darkblue}{rgb}{0,0,0.4}
\definecolor{links}{rgb}{0.47, 0.27, 0.23}

\hypersetup{
    colorlinks, linkcolor={links},
    citecolor={links}, urlcolor={links}
}

\newcommand{\db}{\textcolor{darkblue}}
\newcommand{\Kappa}{K}
\newcommand{\df}[1]{\emph{\db{#1}}}
\newcommand{\m}[1]{\db{$#1$}}
\newcommand{\N}{\mathbb{N}}
\newcommand{\NI}{\N_\infty}
\newcommand{\WW}{\operatorname{W}}
\newcommand{\R}{\mathbb{R}}
\newcommand{\U}{\mathcal{U}}
\newcommand{\V}{\mathcal{V}}
\newcommand{\W}{\mathcal{W}}
\newcommand{\Zero}{\mathbbm{O}}
\newcommand{\One}{\mathbbm{1}}
\newcommand{\Two}{\mathbbm{2}}
\newcommand{\Id}{\operatorname{Id}}
\newcommand{\refl}{\operatorname{refl}}
\newcommand{\J}{\operatorname{J}}

\begin{document}

\maketitle

\paragraph{Our univalent type theory.}
We work with an
\href{https://www.sciencedirect.com/science/article/pii/S0049237X08719451}{intensional
  Martin-L\"of type theory} with an empty type \m{\Zero}, a one
element type \m{\One}, a type \m{\Two} with points \m{0} and \m{1}, a
type \m{\N} of natural numbers, and type formers \m{+} (disjoint sum),
\m{\Pi} (product), \m{\Sigma} (sum), \m{\WW} types, and \m{\Id}
(identity type), and a hierarchy of type universes closed under them,
ranged over by \m{\U,\V,\W}. On top of that we add Voevodsky's
\href{https://homotopytypetheory.org/2018/03/07/a-self-contained-brief-and-complete-formulation-of-voevodskys-univalence-axiom/}{univalence}
axiom and a \href{https://homotopytypetheory.org/book}{propositional
  truncation} axiom.

A formal version of the development discussed here is available in our
github repository
\href{https://github.com/martinescardo/TypeTopology}{TypeTopology}, in
\href{https://wiki.portal.chalmers.se/agda/pmwiki.php}{Agda} with the
option
\href{https://www.cambridge.org/core/journals/journal-of-functional-programming/article/eliminating-dependent-pattern-matching-without-k/4BC4EA2D02D801E5ABED264FE5FB177A}{--without-K}. We
are considering porting this to
\href{https://homotopytypetheory.org/2018/12/06/cubical-agda/}{cubical
  Agda}, so that no axioms are used and our results get a
computational interpretation.

By a \df{proposition} we mean a type with at most one element (any two of
its elements are equal in the sense of the identity type). The
\df{existential quantification} symbol \m{\exists} denotes the
propositional truncation of \m{\Sigma}. We denote the identity type
\m{\Id \, X \, x \, y} by \m{x = y} with \m{X} elided.  We assume the
notation and terminology of the
\href{https://homotopytypetheory.org/book/}{HoTT Book} unless
otherwise stated.

\paragraph{Compact types.}
We consider three notions of exhaustively searchable type.  We say
that a type \m{X} is \df{compact}, or sometimes
\m{\Sigma}-\df{compact} for emphasis, if the type \m{\Sigma (x : X), p
  \, x = 0} is decidable for every \m{p : X \to \Two}, so that we can
decide whether \m{p} has a root. We also consider two
successively weaker notions, namely \m{\exists}-\df{compactness} (it
is decidable whether there is an unspecified root) and
\m{\Pi}-\df{compactness} (it is decidable whether all points of \m{X} are
roots), obtained by replacing \m{\Sigma} by \m{\exists} and
\m{\Pi} in the definition of compactness.

For the model of simple types consisting of
\href{https://www.springer.com/gb/book/9783662479919}{Kleene--Kreisel
  spaces}, these notions of compactness agree and
\href{https://lmcs.episciences.org/693}{coincide with topological
  compactness} under classical logic, but we reason constructively
here, meaning that we don't invoke (univalent) choice or excluded
middle.

Finite types of the form \m{\One + \One + \cdots + \One} are clearly
compact. The compactness of \m{\N} is
\href{https://plato.stanford.edu/entries/mathematics-constructive/}{LPO}
(limited principle of omniscience), which happens to be equivalent to
its \m{\exists}-compactness, and its \m{\Pi}-compactness is equivalent
to
\href{https://journals.openedition.org/philosophiascientiae/406}{WLPO}
(weak LPO), and hence all forms of compactness for \m{\N} are not
provable or disprovable in our classically/constructively-neutral
foundation.

An example of an \df{infinite} compact type is that of conatural
numbers, \m{\NI}, also known as the \df{generic convergent sequence}
(this was presented in
\href{https://sites.google.com/site/types2011/program}{Types'2011} in
Bergen).  This type, the final coalgebra of \m{- + \One}, is not
directly available in our type theory, but can be constructed as the
type of
\href{https://www.cambridge.org/core/journals/journal-of-symbolic-logic/article/infinite-sets-that-satisfy-the-principle-of-omniscience-in-any-variety-of-constructive-mathematics/0D204ADE629B703578B848B8573FC83D}{decreasing
  infinite binary sequences}.

We are able to construct plenty of \df{infinite} compact types, and it
turns out they all can be equipped with well-orders making them into
ordinals.

\paragraph{Ordinals.}
An \href{https://homotopytypetheory.org/book/}{ordinal} is a type
\m{X} equipped with a proposition-valued binary relation
\m{{-}\,{<}\,{-} : X \to X \to \U} which is transitive, well-ordered
(satisfies transfinite induction), and extensional (any two elements
with the same predecessors are equal). The HoTT Book additionally
requires the type \m{X} to be a set, but we show that this follows
automatically from extensionality. For example, the types of natural
and conatural numbers are ordinals. By univalence, the type of
ordinals in a universe \href{https://homotopytypetheory.org/book/}{is
  itself an ordinal} in the next universe, and in particular is a set.

Addition is implemented by the type former \m{-+-}, and multiplication
by the type former \m{\Sigma} with the lexicographic order. The
compact ordinals we construct are, moreover, \df{order-compact} in the
sense that a minimal element of \m{\Sigma (x : X), p\,x = 0} is found,
or else we are told that this type is empty.  Additionally, we have a
selection function of type \m{(X \to \Two) \to X} which gives the
infimum of the set of roots of any \m{p : X \to \Two}, and in
particular our compact ordinals have a top element by considering \m{p
  = \lambda x. 1}.

\paragraph{Discrete types.}
We say that a type is \df{discrete} if it has decidable equality.

\paragraph{Totally separated types.}
It may happen that a non-trivial type has no nonconstant function into
the type \m{\Two} of booleans so that it is trivially compact. For
example, this would be the case for a type \m{\R} of real numbers
under Brouwerian continuity axioms. Under such axioms, such types are
compact, but in a uninteresting way. We say that a type is \df{totally
  separated}, again borrowing a
\href{https://www.encyclopediaofmath.org/index.php/Totally_separated_space}{terminology
  from topology}, if the functions into the booleans separate the
points, in the sense that it satisfies the \df{boolean Leibniz principle}
that says that any two points that satisfy the same boolean-valued
predicates are equal. Such a type is automatically a set, or a
\m{0}-groupoid, in the sense of univalent mathematics. We construct a
totally separated reflection for any type, and show that a type is
compact, in any of the three senses, if and only if its totally
separated reflection is compact in the same sense.

\paragraph{Interplay between the notions.}
We show that compact types, totally separated types, discrete types
and function types interact in very much the same way as the
topological concepts with the same names, where arbitrary functions in
type theory play the role of continuous maps in topology, \df{without}
assuming Brouwerian continuity axioms. For instance, if the types \m{X
  \to Y} and \m{Y} are discrete then \m{X} is \m{\Pi}-compact, and if
\m{X \to Y} is \m{\Pi}-compact, and \m{X} is totally separated
and~\m{Y} is discrete, then \m{X} is discrete, too. The simple types
are all totally separated, which agrees with the situation with
Kleene--Kreisel spaces, but it is easy to construct types which fail
to be totally separated (e.g.\ the homotopical circle) or whose total
separated gives a constructive taboo (e.g.\ \m{\Sigma (x : \NI), x =
  \infty \to \Two}, where the idea is to get two copies of the point
at infinity).

\paragraph{Notation for discrete and compact ordinals.}
We define infinitary ordinal codes, or expressions, similar to the
so-called
\href{http://www.cse.chalmers.se/~coquand/ordinal.ps}{Brouwer
  ordinals}, including one, addition, multiplication, and countable
sum with an added top point.

We interpret these trees in two ways, getting discrete and compact
ordinals respectively. In both cases, addition and multiplication
nodes are interpreted as ordinal addition and multiplication. But in
the countable sum with a top point, the top point is added with
\m{-+\One} in one case, and so is isolated, and by a limit-point
construction in the other case (given our sequence \m{\N \to \U} of
types, we extend it to a family \m{\NI \to \U} so that it
maps~\m{\infty} to a singleton type, by a certain type
\href{https://en.wikipedia.org/wiki/Injective_object}{injectivity}
construction, and then take its sum).

We denote the above interpretations of ordinal notations \m{\nu} by
\m{\Delta_\nu} and \m{\Kappa_\nu}. The types in the image of
\m{\Delta} are discrete and retracts of \m{\N}, and those in the image
of \m{\Kappa} are compact, totally separated and retracts of the
Cantor type \m{\N \to \Two}.  Moreover, there is an order preserving
and reflecting embedding \m{\Delta_\nu \to \Kappa_\nu}, which is an
isomorphism if and only if \db{LPO} holds, but whose image always has
empty complement for all ordinal notations \m{\nu}. An example of such
a situation is the evident embedding \m{\N + \One \to \NI} (this
inclusion is merely a monomorphism, rather than a topological
embedding, in topological models -- the word \df{embedding} in
univalent mathematics refers to the appropriate notion for
\m{\infty}-groupoids, which in this example are \m{0}-groupoids).  By
transfinite iteration of the countable sum, one can get
\href{http://www.cs.swan.ac.uk/~csetzer/articles/weor0.pdf}{rather
  large} compact ordinals.

\end{document}